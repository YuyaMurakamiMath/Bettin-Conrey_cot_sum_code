% --------------------------------------------------------------------------
%		Document style
% --------------------------------------------------------------------------

\documentclass[11pt]{amsart}
\usepackage[marginparwidth=0pt,margin=20truemm]{geometry} % margin

% --------------------------------------------------------------------------
%		Packages
% --------------------------------------------------------------------------

\usepackage{amssymb}
\usepackage{amsmath}
\usepackage{amsthm}
\usepackage{color}
\usepackage{graphicx}
%\usepackage[hiresbb,dvipdfmx]{graphicx}
%\usepackage{showkeys}
\newcommand{\wC}{\widetilde{C}}
\newcommand{\Z}{\mathbb{Z}}
\newcommand{\Q}{\mathbb{Q}}
\newcommand{\C}{\mathbb{C}}

\usepackage{url}
\begin{document}
The scripts of this directory is to compute numerical values of $g_n$
in \cite[eq.~(1.2)]{AM} or $\mathcal{I}_n=g_n-\frac{1}{n}$.
As was shown by Bettin--Conrey, it holds that
\[
 g_n=\frac{1}{n(n+1)}+2b_n+2\sum_{j=0}^{n-2}\binom{n-1}{j}b_{j+2}\eqno{(\clubsuit)}
\]
for $n\in\Z_{\geq 2}$, where
\[
 b_k=\frac{B_k\zeta(k)}{k},
\]
$B_k$ is the $k$-th Bernoulli number and $\zeta(s)$ is the
Riemann zeta-function.
According to \cite[the first displayed formula in \S 6]{AM},
we have
\[
 b_{2l}=(-1)^{l-1}\frac{(2\pi)^{2l}B_{2l}^2}{4l\cdot(2l)!}
\]
for $l\in\Z_{\geq 1}$. Since $B_k$ vanishes for odd $k\geq 3$, we find
\[
 |b_k|=\frac{(2\pi)^{k}B_{k}^2}{2k\cdot k!}
\]
for $k\in\Z_{\geq 2}$.

To investigate significant digits of precision needed for computation
of $g_n$, we set
\begin{gather*}
 G_n^{(1)}=\frac{1}{n(n+1)}+2|b_n|+2\sum_{j=0}^{n-2}\binom{n-1}{j}|b_{j+2}|,
\phantom{MM}G_n^{(\infty)}=2\max_{0\leq j\leq n-2}\binom{n-1}{j}|b_{j+2}|.
\end{gather*}
The script \verb|SDP.gp| is to compute $G_n^{(\infty)}$ and $G_n^{(1)}$.
After start Pari/GP, do \verb|\r SDP.gp|. To compute $G_{1001}^{(\infty)}$
(\emph{resp.} $G_{1001}^{(1)}$), type \verb|Ginf(1001)| (\emph{resp.} \verb|Gone(1001)|).
When we calculate $g_n$ up to $1001$, we set significant digits of precision
to much larger than the number of digits of
$G_{1001}^{(1)}$ or $1001\times G_{1001}^{(\infty)}$.

By ($\clubsuit$) we find
\[
 \mathcal{I}_n=-\frac{1}{n+1}+2b_n+2\sum_{j=0}^{n-2}\binom{n-1}{j}b_{j+2}
\]
for $n\in\Z_{\geq 2}$.
Since $b_n$ vanishes for odd $n\geq 3$, we notice
\begin{gather}
 \mathcal{I}_{2l}=-\frac{1}{2l+1}+4l\times b_{2l}+\sum_{j=0}^{l-2}\binom{2l-1}{2j}b_{2(j+1)},\\
\mathcal{I}_{2l+1}=-\frac{1}{2l+2}+\sum_{j=0}^{l-1}\binom{2l}{2j}b_{2(j+1)}
\end{gather}
for $l\in\Z_{\geq 1}$.
The script \verb|main.gp| outputs numerical values of $I_n$
for $2\leq n\leq 1001$ as a vector.
Since $G_{1001}^{(1)}=7.14\ldots\times 10^{1773}$,
we compute $g_n$ with $2000$ decimal digits.
\begin{thebibliography}{99}
 \bibitem{AM}
H.~Akatsuka and Y.~Murakami,
An asymptotic property on a reciprocity law for the Bettin--Conrey
cotangent sum, {\tt arXiv:2402.14216}.
\end{thebibliography}
\end{document}