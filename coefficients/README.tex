% --------------------------------------------------------------------------
%		Document style
% --------------------------------------------------------------------------

\documentclass[11pt]{amsart}
\usepackage[marginparwidth=0pt,margin=20truemm]{geometry} % margin

% --------------------------------------------------------------------------
%		Packages
% --------------------------------------------------------------------------

\usepackage{amssymb}
\usepackage{amsmath}
\usepackage{amsthm}
\usepackage{color}
\usepackage{graphicx}
%\usepackage[hiresbb,dvipdfmx]{graphicx}
%\usepackage{showkeys}
\newcommand{\wC}{\widetilde{C}}
\newcommand{\Z}{\mathbb{Z}}
\newcommand{\Q}{\mathbb{Q}}
\newcommand{\C}{\mathbb{C}}

\usepackage{url}
\begin{document}
The file \verb|coefficients.gp| calculates $\wC_l$ in \cite[eq.~(5.2)]{AM}
up to $100$.
To explain the script, we simply recall the setting.
We set
\[
 \lambda(y)=\frac{1}{e^y-1}-\frac{1}{y}+\frac{1}{2}=\sum_{k=1}^{\infty}\frac{B_k}{(k+1)!}y^k,
\]
where $B_k$ is the $k$-th Bernoulli number.
We define $P_k(T)$ by the generating function via
\[
 e^{-T\lambda(y)}=\sum_{k=0}^{\infty} P_k(T)y^k.
\]
We note $P_0(T)=1$.
According to \cite[Corollaries 5.4 and 5.5]{AM}, we know
$P_k(T)\in\Q[T]$ with degree $k$ and $T^k P_k(T)\in T^2 \Q[T^2]$
for any $k\in\Z_{\geq 1}$.
We also set
\begin{align*}
 \wC_l(T)
&=\sum_{\begin{subarray}{c}
           j,k\geq 0\\
	j+k=l
       \end{subarray}}
(k+1,j)T^k P_k(T)2^{-2j}\\
&=\sum_{k=0}^l (k+1,l-k)T^kP_k(T) 2^{-2(l-k)},
\end{align*}
where $(k+1,j)$ is the Hankel symbol.
We notice $\wC_0(T)=1$.
For $l\geq 1$ we write
\[
 \wC_l(T)=(1,l)2^{-2l}+\sum_{k=1}^l (k+1,l-k)T^kP_k(T) 2^{-2(l-k)}.
\]
We easily see $\wC_l(T)\in\Q[T^2]$ and $\deg_{\Q[T]}\wC_l(T)=2l$.
The numbers $\wC_l$ are given by $\wC_l=\wC_l(2\pi i)$.

To run the script, we firstly change the directory \verb|./coefficients/|.
Then we start Pari/GP and do \verb|\r coefficients.gp|.
Note that the script uses 128 MB.
As a result, output the following three files:
\begin{itemize}
 \item \verb|tCkTout|: save $\wC_l(T)\in\Q[T]$ for $1\leq l\leq 100$ as
a vector.
\item \verb|tCkout_pi|: save $\wC_l(T)$ as $\Q$-linear combinations
of $1,\pi^2,\ldots,\pi^{2l}$ for $1\leq l\leq 100$ as a vector.
Here \verb|p|$=\pi$.
\item \verb|tCkout_num|: save numerical values of $\wC_l$ for
$1\leq l\leq 100$ as a vector.
\end{itemize}
\begin{thebibliography}{99}
 \bibitem{AM}
H.~Akatsuka and Y.~Murakami,
An asymptotic property on a reciprocity law for the Bettin--Conrey
cotangent sum, {\tt arXiv:2402.14216}.
\end{thebibliography}
\end{document}